\section{Michał Dziarkowski}

Zdjęcie:
\begin{figure}[h]
  \includegraphics[width=10cm]{pictures/logo}
  \centering
  \caption{logo Overleaf}
  \label{fig:logo}
\end{figure}

Wyrażenie matematyczne:
\[
\frac{e^x+log_x 8}{\int_1^3 {x^2+2x+1}}
\]



Lista nienumerowana:
\begin{itemize}
  \item[->] Punkt Nienumerowany
  \item[->] Punkt Także Nienumerowany
  \item[->] Kolejny Punkt Nienumerowany
\end{itemize}

Lista numerowana:
\begin{enumerate}
    \item Punkt Pierwszy
    \item Punkt Drugi
    \item Punkt Trzeci
    \item Punkt Czwarty
\end{enumerate}

Tabela:\newline
\begin{table}[htp]
\begin{tabular}{|l|l|l|}
\hline
nazwa   & cena(zł/kg) & stan(szt) \\ \hline
jabłka  & 3           & 120       \\ \hline
banany  & 5           & 33        \\ \hline
gruszki & 3.50        & 40        \\ \hline
\end{tabular}
\label{tab:ceny}
\caption{Tabela przedstawia losowe dane na temat jakichś produktów spożywczych.}
\end{table}

\textbf{Pierwszym poleceniem}, które wykonałem, to \underline{dodanie zdjęcia wraz z opisem} (Patrz: Rysunek \ref{fig:logo})

\textbf{Następnie dodałem} tabelę (Patrz: Tabela \ref{tab:ceny}) , dwie listy, wyrażenie matematyczne i \textit{trochę tekstu}. Przy odpowiednim naciągnięciu definicji, można go zaliczyć jako dwa akapity.
